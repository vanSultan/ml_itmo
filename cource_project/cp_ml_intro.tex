\newpage
\section{Introduction}\label{sec:intro}

The easy access and exponential growth of information available on social media has complicated the distinction between false and true information.
The easy dissemination of information through sharing has exacerbated the exponential growth of falsification.
At stake is also the credibility of social media where false information is disseminated.
Thus, the research challenge is to automatically verify information, namely its source, content and publisher, to classify it as false or true.

Machine learning has played a vital role in classifying information, albeit with some limitations.
This paper discusses various machine learning approaches to detect fake and fabricated news.
The limitation of such approaches and improvisations by implementing deep learning is also discussed.

The term "\textit{fake news}", a Collins' Word of the Year 2017, has gained popularity as a result of Donald Trump's election campaign, which refers to deliberately false and inaccurate mainstream media coverage.
Interestingly, deliberately false information is much more popular with new media users: a Twitter study concluded that fake news stories are 70\% more likely to be retweeted, while true stories are on average six times less likely to be viewed than fake news.
Creating a fake news story that is likely to be popular is much easier than finding a really worthwhile true news story.
In addition, given that despite attempts by states to combat the flow of false information, it is impossible to fully control new media, unlike traditional media, fake news is published and distributed freely in the media space, contributing to the increasing gap between hyper-reality and physical reality, as they form media models that distort reality.
There are a number of classic tricks to draw attention to a false infopoint and hook the audience, such as the use of children in negative connotations (death of a child, illness of a child, etc.), fabricated visual evidence and scientific sensationalism, use of false testimony, demonisation of the enemy, etc.
All of these techniques have one common flaw: they require a specific context for their use.