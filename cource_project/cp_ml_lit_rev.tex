\newpage
\section{Literature review}\label{sec:lit_rev}

Various approaches to detecting fake news have been proposed over the recent years.
Before going on to an overview of methods to combat the spread of fake news, it is necessary to define what fake news is and what it can be.

The article "\textit{Fake News: A Definition}" denotes fake news as a sociological phenomenon~\cite{0}.
According to the article, fake news is false, often sensational, information disseminated under the guise of news reporting.

In the article "\textit{Deception detection for news: three types of fakes}" by V. L. Rubin, Y. Chen, and N. J. Conroy presents the following classification of fake news~\cite{1}.

The first type, serious fabrications, includes publications in tabloids.
These outlets use eye-catching headlines, exaggerations, scandal-mongering, or sensationalism to increase traffic or profits.
Yellow press focuses on breaking news, especially crime, gossip columns and astrology.

The second type, hoaxes, is social media or journalistic news.
Unlike jokes, hoaxes are "relatively complex and large-scale fabrications" that may involve deceptions beyond mere gambling and "cause material damage or harm to the victim".

A third type, humorous news, includes news sites that specialise in fake information and do not pretend to be a true source, games to identify fake news, and satire and parody.
The characteristics of this type include the fact that, due to their humorous tone, they can be easily identified.

In the article "\textit{Fake News Detection Using Machine Learning approaches: A systematic Review}", another classification of fake news is highlighted~\cite{2}:

\begin{enumerate}
    \item Visual-based: These fake news posts use graphics a lot more in as content, which may include morphed images, doctored video, or combination of both.
    \item User-based: This type of fabricated news is generated by fake accounts and is targeted to specific audience which may represent certain age groups, gender, culture, political affiliations.
    \item Knowledge-based: these types posts give scientific explanation to the some unresolved issues and make users to believe it is authentic.
    For example natural remedies of increased sugar level in human body.
    \item Style-based posts are written by psedojournalists who pretend and copy style of some accredited journalists.
    \item Stance-based: It actually is representation of truthful statements in such a way which changes its meaning and purpose.
\end{enumerate}

The article "\textit{Media-Rich Fake News Detection: A Survey}" provides sources of fake news~\cite{3}.

Standalone website:
\begin{enumerate}
    \item Popular news sites.
    \item Blog sites: Blog sites are big on user-generated content and heavily rely on unsupervised content, also considered the best place to get wrong information.
    \item Media sites.
\end{enumerate}

Social media: Sharing is the most common way of circulating the content on these sites.
More than 70\% of its users use them for their daily news source.
\begin{enumerate}
    \item Facebook (Status \textemdash Wall Posts \textemdash Dedicated Page \textemdash Ad): Users can make a Facebook wall post and/or create Facebook pages and produce/share content using these pages.
    Better yet, it is concerning to see Facebook allowing users to create paid ads for pretty much any post, which could very well be fake news and can reach larger audience.
    \item Twitter (Tweet \textemdash Re-Tweet): Twitter is also a social media site, allows you to create a tweet (limited character) and retweet (share a popular tweet with other).
    \item Emails: Emails (news) are also a great way for consumers to receive news, it is really challenging to validate the authenticity of news emails.
    \item Broadcast networks (PodCast): Podcasts are an audio multimedia category, very small number of users still use this service and consume their news.
    \item Radio service: Radio talk shows are a popular source of news and it is challenging to validate the truthfulness of the audio.
\end{enumerate}

The article "\textit{Fake news detection on social media: A data mining perspective}" provides basic methods for analyzing fake news~\cite{9}:
\begin{enumerate}
    \item Linguistic Features based methods.
    \item Deception Modeling based methods.
    \item Clustering based methods.
    \item Predictive Modeling based methods.
    \item Content Cues based methods.
    \item Non-Text Cues based methods.
\end{enumerate}

The article "\textit{Automatic deception detection: Methods for finding fake news}" by Conroy, N. J., Rubin, V. L., \& Chen, Y. (2015, November) describes the typologies of several varieties of credibility assessment methods arising from two main categories: linguistic replication approaches and network analysis approaches~\cite{4}.
A fake news detection method that is a hybrid of these two approaches is also presented.

The article "\textit{Detecting Fake News using Machine Learning: A Systematic Literature Review}" provides the most common methods of machine learning~\cite{8}.
These methods include the following approaches.

\textit{Support Vector Machine}: This algorithm is mostly used for classification.
This is a supervised  machine learning algorithm that learns from the labeled data set.
Researchers in (Singh et al., 2017) used various classifiers of machine learning and the support vector machine have given them the best results in detecting the fake news.

\textit{Naive Bayes}: Naive Bayes is also used for the classification tasks.
This can be used to check whether the news is authentic or fake.
Researchers in (Pratiwi et al., 2017) used this classifier of machine learning to detect the false news.

\textit{Logistic Regression}: This classifier is used when the value to be predicted is categorical.
For example, it can predict or give the result in true or false.
Researchers in (Kaur et al., 2020) have used this classifier to detect the news whether it is true or fake.

\textit{Random Forests}: In this classifier, there are different random forests that give a value and a value with more votes is the actual result of this classifier.
In (Ni et al., 2020) researchers have used different machine learning classifiers to detect the fake news.
One of these classifiers is the random forest.

\textit{Recurrent Neural Network}: This classifier is also helpful for detecting the fake news.
Researchers in (Jadhav \& Thepade, 2019) have used the recurrent neural network to classify the news as true or false.

\textit{Neural Network}: There are different algorithms of machine learning that are used to help in classification problems.
One of these algorithms is the neural network.
Researchers in (Kaliyar et al., 2020) have used the neural network to detect the fake news.

\textit{K-Nearest Neighbor}: This is a supervised algorithm of machine learning that is used for solving the classification problems.
This stores the data about all the cases to classify the new case on the base of similarity.
Researchers (Kesarwani et al., 2020) have used this classifier to detect fake news on social media.

\textit{Decision Tree}: This supervised algorithm of machine learning can help to detect the fake news.
It breaks down the dataset into different smaller subsets.
Researchers in (Kotteti et al., 2018) have used different machine learning classifiers and one of them is the decision tree.
They have used these classifiers to detect the fake news.

The article "\textit{A Benchmark Study on Machine Learning Methods}" for Fake News Detection compares deep learning methods and neural networks with the classical machine learning methods presented above~\cite{7}.

In article "\textit{Automatic Online Fake News Detection Combining Content and Social Signals}" is presented by the novel ML fake news detection method which uses combination of news content and social context features~\cite{5}.
By combining these approaches, the authors of the article managed to increase the accuracy of the method.

In the article "\textit{Some Like it Hoax: Automated Fake News Detection in Social Networks}", the authors hypothesize the relationship between the many users who interact with the news and the fact that the news is fake or not~\cite{6}.
